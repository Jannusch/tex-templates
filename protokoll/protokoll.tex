\documentclass{fsrprotokoll}


% Das %dontbuild muss _allein_ auf der ersten Zeile stehen. 
% Sobald es entfernt ist, wird ein Protokoll mit dem Namen 
% "Protokoll DD.MM.YYYY" automatisch gebaut und auf den FTP 
% geschoben. Das Protokoll mit dem neusten Datum wird
% außerdem unter latest verlinkt.


\begin{document}

%%!sitzungsleitung,protokoll: Das Protokoll ist fertig und kann unterschrieben werden! :)

\date{25.05.2020}  % DD.MM.YYYY
\attendeesNumber{13}  % Anzahl anwesender Mitglieder
\chairperson{Jasmin Dettelbach}
\clerks{Anita Fritzsche, Jannusch Bigge, Pascal Scholz}
\meetingStart{18:47}  % HH:MM
\meetingEnd{19:25}  % HH:MM
% Liste anwesender Mitglieder
\attendees{Anita Fritzsche, Benjamin Klahn,  Emmanuel Diehl, Franz Rodestock, Jakob Krebs, Jannusch Bigge, Jasmin Dettelbach, Marcel Legler, Pascal Scholz, Rebecca Uecker, Robert Glöckner, Robert Peine, Teodora Ivoniciu}
\excused{}
\unexcused{}
% Liste Ruhender
\resting{Eddy Loose, Hendrik Appel, Jakob Behner, Mohd Faraz Shaikh, Thomas Birkenheuer}
% Liste anwesender Gäste
\guests{Christina Ulonska, Lucas Hecht, Mathias Stuhlbein, Patrik Phan,  Dirk Legler, Sascha Peukert, Tim Haering}

\maketitle

Die  Sitzungsleitung eröffnet die Sitzung. Sie findet gemäß Beschluss  2020/27 i.V.m Beschluss 2020/30 aufgrund der aktuellen  Coronavirus-Pandemie via BigBlueButton digital statt. Dies wurde 48 Stunden vor Sitzungsbeginn u.a. auf der Website bekanntgegeben. Dort wurde auch der Zugriffslink veröffentlicht.

\quorum{13}{13} % Feststellung der Beschlussfähigkeit {#Anwesende}{#NichtRuhende}

\section{Protokollkontrolle und Überprüfung gefasster Beschlüsse}
Es gibt nichts zu beanstanden, das Protokoll vom 18.05.2020 wird bestätigt. 

\section{Berichte der Ämter}
\subsection{Strukturer\_innen}
Jakob K. berichtet, dass weiterhin jemand für die StuKo DSE gesucht wird und wir wirklich langsam jemanden entsenden sollten. Interessierte können sich gern melden.

\section{Berichte der Arbeitsgruppen}

\subsection{AG Öffentlichkeitsarbeit}

Patrik berichtet, dass Samstag auf Instagram die Vorstellung über die AG Nachhaltigkeit in den sozialen Medien veröffentlicht wurde. Es gibt Neuigkeiten zur neuen Website, es wurden technische Details geklärt. Patrik stellt den aktuellen Stand vor.

\appear{Lucas Hecht}

Auf der neuen Webseite wird es Unterseiten zu den einzelnen Teilen des Fachschaftrat geben. Patrik bittet darum, dass sich hier Freiwillige finden, die kurze Texte und Beschreibungen zuarbeiten.

Franz gibt positives Feedback. Jasmin fragt, ob sich die Lesbarkeit verbessert hat, da uns vor kurzem eine Anfrage dazu erreicht hat.
Patrik erklärt, dass es einen Button am unterem Ende der Webseite geben wird, mit dem sich ein Modus für Barrierefreiheit aktivieren lässt. Weiterhin wird es einen Darkmode geben,  welcher je nach Geräteeinstellungen aktiviert wird.
Christina fragt, ob es die Website auch auf Englisch gibt. Er bejaht dies.

\section{Berichte aus den Gremien}
Es gibt nichts zu berichten.

\section{Stand der Veranstaltungen}
Jasmin berichtet, dass das Robolab wieder angelaufen ist. Es scheint bisher gut zu laufen.

\section{Richtlinien für @extern}
Anita berichtet, dass sie zumsammen mit der Hilfe von Patrik eine Richtlinie ausgearbeitet hat, in welcher geregelt werden soll, welche Mails auf extern@ akzeptiert werden. Dies ist nötig, da sie häufiger bei der Moderation mit den Antragsstellern diskutieren muss, weswegen Mails abgelehnt wurden. Es wird beschlossen, dass sich alle die Richtlinien durchlesen sollten. Auf der nächsten Sitzung sollen sie dann beschlossen werden, nachdem Änderungen eingebracht werden konnten.

\section{Berichte von der KIF 48.0}
Jakob K. berichtet, dass vergangenes Wochenende die KIF stattgefunden hat. Christina sowie einige ehemalige FSR-Mitgleider haben daran teilgenommen. Die KIF hat virtuell stattgefunden. Die einzelnen Arbeitskreise fanden in BigBlueButtom Meetings satt und auch die Plena wurden digital gehalten.
Jakob K. meint, es wurden Punkte notiert die man gut fand und so auch in einer eventuellen digitalen ESE umsetzen könnte.

Es gab desweitern Resolutionen, welche sich mit Prüfungen im Rahmen der Corona Pandemie beschäfftigen. Er schlägt vor die aktuelle Situation an der TU Dresden mit den beschriebenen Vorschlägen zu vergleichen und bei Unstimmigkeiten diese im Fakultätsrat anzusprechen.
Jakob K. berichtet von einer Resolution, die sich damit beschäftigt, dass allen Menschen die Partizipation an der Lehre offenstehen und es nötige Nachteilsausgleiche geben sollte.
Weiterhin gab es eine Resolution zu Videokonferenzsystemen, die sich im Kern gegen Zoom richtet und Empfehlungen für Alternativen ausspricht.
Die Resolutionen können alle hier\footnote{https://wiki.kif.rocks/wiki/Portal:Resolutionen} eingesehen werden.
Jakob K. fragt ob andere Teilnehmer mehr berichten können.

Dirk berichtet daraufhin weiter, dass Dortmund leider nicht die beste Internet-Infrastruktur hat, so dass die Meetings auch von Ilmenau und Göttingen gehostet wurden. Die Teilnehmer waren dabei rücksichtsvoll und haben sich einander auch in größeren Plena gegenseitig aussprechen lassen.
Er merkt an, das dies sehr gut zeigt, dass es möglich ist große Veranstaltungen online zu veranstalten, sowohl technisch, als auch in Hinblick auf Gesprächsführung.
Sascha hat eine Anmerkung zum Pinguinfrühstück eingebracht.

\section{Sitzung am 01.06.2020}
Jasmin fragt, ob wir am nächsten Montag eine Sitzung abhalten wollen. Franz denkt, dass man die Sitzung aufgrund mangelnder TOPS ausfallen lassen könnte.
Jasmin fragt, ob diese Woche noch wichtige Gremien tagen und ob es nötig wäre, diese Treffen am Montag zu besprechen. Falls dies nicht der Fall ist, schlägt sie vor, die Sitzung in der nächsten Woche ausfallen lassen.
Christina denkt, dass die Sitzung ausfallen kann, aber würde gerne noch einmal auf die Wahl der StuKo DSE aufmerksam machen.

 \begin{vote-two-thirds}
   \voteMoney{}  % {Betrag (ohne \EUR)} nur bei Finanzantrag
   \voteNumber{32}  % {#Antrag}
   \voteText{Der FSR Informatik möge beschließen, dass die Sitzung am 01.06.2020 aufgrund des Feiertags ausfällt.}
   \voting{8}{1}{3}  % {#Dafür}{#Dagegen}{#Enth.}, leer lassen für 'ohne Gegenrede angenommen'
\end{vote-two-thirds}
% Die Klasse wird hier ausspucken, dass der Antrag angenommen wurde. Der Antrag wurde aber abgelehnt, da eine 2/3-Mehrheit (= 9 Dafür-Stimmen) notwendig gewesen wäre. Man müsste also mal die Klassen anpassen :P Boar -_- 

\section{Sonstiges}
Es gab leichte Unklarheiten bei der Auswertung des Ergebnisses. Robert P. fragt, ob die Abstimmung wiederholt werden soll. Niemand will die Abstimmung wiederholen. Die Sitzung nächste Woche findet statt.

\subsection{Emoji der Woche}
Das Emoji der Woche ist das Kaktus-Emoji. Es wurde mit der Unicode Version 6.0 veröffentlicht. Diese wurde in 2010 spefiziert und dann im Jahre 2015 veröffentlicht. Der Code des Kaktus-Emoji ist (U+1F335).

Patrik zeigt nochmal zwei Screenshots der Webseite, die die barrierefreie Version und den Darkmode zeigen.

Die Sitzungsleitung beendet die Sitzung um 19:25 Uhr.
\section*{English Summary}
\subsection{Reports of persons resposible of inner structure}
The seat in the study comission DSE is still not staffed.

\subsection{Report of AG Öffentlichkeitsarbeit (public relations activities)}
An Instram story was published, which introduced the AG Nachhaltigkeit (sustainability). Moreover prototype screenshots of our new website were shown. The new website will provide features like a dark mode and a mode for high contrast. Also, an english version of the website will be offered.

\subsection{Events}
The robolab has started again. It originally was ment to fill the first four weeks of the lecture free time but than had to be paused because of the corona virus.

\subsection{Guidelines regarding e-mail distribution list extern@}
Guidelines regarding e-mail distribution list extern@ were shown. All attendees were asked to read these guidelines, so the can be discussed in the next meeting. The distribution list extern@ is used for distributing external job offers. These offers are checked before distributing and when they were declined, sometimes arguments took place between us and the external companies, because no clear and visible guidelines existed.

\subsection{Report from KIF 48.0}
The KIF is an national meeting of student coucils of computer science. Digital teaching as well as holding events in a digital way were discussed. One Resolutions was decieded, which critizes the use of zoom and provides suggestions for alternatives. Another resolution was decided which expresses the right for each student to take part in digital teaching and how compensatory measures could look. The event was also held in a digital way and had a huge amount of attendees, which shows that these kind of events can actually be done in a digital way.

\subsection{Meeting on 1st June 2020}
Because the 1st June is holliday, it was asked if the meeting should be done. A decision which asked for cancelling the meeting was rejected.

\subsection{Misc}
The emoji of the week is the cactus emoji. It's code is U+1F335 and it's part of the unicode standard since version 6.0.

\signature


% % Die Reihenfolge muss beibehalten werden.
% \begin{vote}
% %   \voteMoney{}  % {Betrag (ohne \EUR)} nur bei Finanzantrag
%   \voteNumber{}  % {#Antrag}
%   \voteText{}
%   \voteReason{}
%   \voting{}{}{}  % {#Dafür}{#Dagegen}{#Enth.}, leer lassen für 'ohne Gegenrede angenommen'
%   \voteComment{}  % optional
% \end{vote}

% \begin{poll}
%   \pollText{}
%   \voting{}{}  % {#Dafür}{#Dagegen}
% \end{poll}

% \appear{eintreffende Person(en)}
% \away{Sitzung verlassende Person(en)}
% \EUR{#Betrag} <- Bitte ausschließlich so Geldbeträge in Euro angeben

% KEIN \emph, \textbf, o.ä. in (sub)section titles! (breaks accessibilityMeta package)
%            Bitte vorzugsweise \section{} und \subsection{} benutzen.      Außerdem       bitte den Gebrauch  von line breaks (\\) einschränken     und  sollte       tatsächlich Bedarf nach mehr Struktur bestehen,      vorzugsweise eine     leere   Zeile bzw. \par verwenden.

% Sollten beim Bauen Fehler auftreten, wird eine Mail an
% strukturer@ifsr.de geschickt. Bei erfolgreichem Build wird das
% Protokoll direkt an den Simplexdrucker gesendet.

% Zeilen die mit %%!<usernamelist>: Todotext beginnen enthalten 
% Todos eine Beispielzeile sieht folgendermassen aus:
% %!marius,frank,strukturer:Organisation von Event nicht vergessen
% (Leerzeichen zwischen den beiden % darf nicht drin sein!!)
% TODOs werden automatisch beim Bauen an die aufgezählten 
% namen@ifsr.de versendet

% Bei einem geheimen Sitzungsteil bitte Pad 1340 (Protokollvorlage geheimer
% Sitzungsteil) beachten bzw. verwenden.

\end{document}





