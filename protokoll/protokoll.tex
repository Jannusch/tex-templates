%dontbuild
\documentclass{fsrprotokoll}


% Das %dontbuild muss _allein_ auf der ersten Zeile stehen.
% Sobald es entfernt ist, wird ein Protokoll mit dem Namen
% "Protokoll DD.MM.YYYY" automatisch gebaut und auf den FTP
% geschoben. Das Protokoll mit dem neusten Datum wird
% außerdem unter latest verlinkt.


\begin{document}

%%!sitzungsleitung,protokoll: Das Protokoll ist fertig und kann unterschrieben werden! :)

\date{}  % DD.MM.YYYY
\attendeesNumber{}  % Anzahl anwesender Mitglieder
\chairperson{}
\clerks{}
\meetingStart{}  % HH:MM
\meetingEnd{}  % HH:MM
% Liste anwesender Mitglieder
\attendees{}
\excused{}
\unexcused{}
% Liste Ruhender
\resting{}
% Liste anwesender Gäste
\guests{Matthias Stuhlbein}

\maketitle

Die Sitzungsleitung eröffnet die Sitzung.

\quorum{}{} % Feststellung der Beschlussfähigkeit {#Anwesende}{#NichtRuhende}

\section{Protokollkontrolle und Überprüfung gefasster Beschlüsse}
% Es gibt nichts zu beanstanden, das Protokoll vom XX.YY.ZZZZ wird bestätigt.

%\section{Gäste}
\section{Berichte der Ämter}
\subsection{Sprecher\_innen}

\subsection{Struktur}

\subsection{Finanzer\_innen}

\section{Berichte der Arbeitsgruppen}

\subsection{AG Lehre}

\subsection{AG Nachhaltigkeit}

\subsection{AG Öffentlichkeitsarbeit}

% \section{Berichte aus den Gremien}
% \subsection{Studierendenrat (StuRa)}

% \subsection{Fakultätsrat (FR)}

\section{Stand der Veranstaltungen}

\section{Weitere anstehende Termine}

\section{Sonstiges}

\signature


% % Die Reihenfolge muss beibehalten werden.
% \begin{vote}
%   \voteMoney{}  % {Betrag (ohne \EUR)} nur bei Finanzantrag, sonst leer lassen
%   \voteNumber{}  % {#Antrag}
%   \voteText{}
%   \voteSubmitter{}  % optional
%   \voteReason{}
%   \voting{}{}{}  % {#Dafür}{#Dagegen}{#Enth.}, leer lassen für 'ohne Gegenrede angenommen'
%   \voteComment{}  % optional
% \end{vote}

% \begin{vote-two-thirds}
%   \voteMoney{}  % {Betrag (ohne \EUR)} nur bei Finanzantrag, sonst leer lassen
%   \voteNumber{}  % {#Antrag}
%   \voteText{}
%   \voteSubmitter{}  % optional
%   \voteReason{}
%   \voting{}{}{}  % {#Dafür}{#Dagegen}{#Enth.}, leer lassen für 'ohne Gegenrede angenommen'
%   \voteComment{}  % optional
% \end{vote-two-thirds}

% \begin{poll}
%   \pollText{}
%   \voting{}{}  % {#Dafür}{#Dagegen}
% \end{poll}

% \appear{eintreffende Person(en)}
% \away{Sitzung verlassende Person(en)}
% \EUR{#Betrag} <- Bitte ausschließlich so Geldbeträge in Euro angeben

% KEIN \emph, \textbf, o.ä. in (sub)section titles! (breaks accessibilityMeta package)
% Bitte vorzugsweise \section{} und \subsection{} benutzen. Außerdem bitte den
% Gebrauch  von line breaks (\\) einschränken und sollte tatsächlich Bedarf nach
% mehr Struktur bestehen, vorzugsweise eine leere Zeile bzw. \par verwenden.

% Sollten beim Bauen Fehler auftreten, wird eine Mail an strukturer@ifsr.de
% geschickt. Bei erfolgreichem Build wird das Protokoll direkt an den
% Simplexdrucker gesendet.

% Zeilen die mit %%!<usernamelist>: Todotext beginnen enthalten Todos eine
% Beispielzeile sieht folgendermassen aus:
% %!marius,frank,strukturer:Organisation von Event nicht vergessen
% (Leerzeichen zwischen den beiden % darf nicht drin sein!!)
% TODOs werden automatisch beim Bauen an die aufgezählten namen@ifsr.de
% versendet

% Bei einem geheimen Sitzungsteil bitte Pad 1340 (Protokollvorlage geheimer
% Sitzungsteil) beachten bzw. verwenden.

\end{document}

